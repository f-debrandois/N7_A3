\documentclass[12pt,a4paper]{article}
\usepackage[utf8]{inputenc} 
\usepackage[T1]{fontenc}		       
\usepackage{lmodern}			       
\usepackage{babel} 
\usepackage{amsmath}
\usepackage{amsfonts}
\usepackage{amssymb}
\usepackage{graphicx}
\usepackage{xcolor}
\usepackage{mathtools}
\usepackage{fancyhdr}
\usepackage{enumitem}
\usepackage{tcolorbox}
\usepackage{stmaryrd}
\usepackage{dsfont}
\usepackage{pgf, tikz}
\usetikzlibrary{shapes.misc}
\usepackage[linesnumbered,ruled,vlined]{algorithm2e}
\usepackage[text={15cm,24.5cm},centering]{geometry}


% Définir le texte affiché en fin de page
\pagestyle{fancy}
\fancyhf{}  % Clear the default headers and footers
\rfoot{\hrule
    \vspace{0.3cm}
    \noindent\textsf{Félix de Brandois}
    \hfill \thepage
}
\renewcommand{\headrulewidth}{0pt}

% Style de l'entete
\newcommand{\entete}{
    \noindent\textbf{INSA - ModIA, 5$^e$ année.}
    \hfill \textbf{Années 2024-2025}
    
    \begin{center}
        \textbf{\LARGE Résolution de systèmes linéaires issus d'EDP}
    \end{center}
}


% Définir la fonction pour créer une boîte de propriété
\newcommand{\propriete}[2]{%
    \begin{tcolorbox}[colback=white,colframe=green!25!white,title=\textbf{Propriété #1}, coltitle=black]
        #2
    \end{tcolorbox}
}

% Définir la fonction pour créer une boîte de définition
\newcommand{\definition}[2]{%
    \begin{tcolorbox}[colback=white,colframe=blue!25!white,title=\textbf{Définition #1}, coltitle=black]
        #2
    \end{tcolorbox}
}

% Définir la fonction pour créer une boîte de théorème
\newcommand{\theoreme}[2]{%
    \begin{tcolorbox}[colback=white,colframe=red!25!white,title=\textbf{Théorème #1}, coltitle=black]
        #2
    \end{tcolorbox}
}

% Définir la fonction pour créer une boîte de remarque
\definecolor{customRed}{RGB}{150, 30, 30}
\newcommand{\remarque}[1]{%
    \leftline{\noindent
    \textcolor{customRed}{\vrule width 3pt}\hspace{10pt}%
    \parbox{0.9\textwidth}{%
        \textbf{Remarque :}
        #1
    }}
    \vspace{10pt}
}

% Définir la fonction pour créer une boîte de preuve
\newcommand{\preuve}[1]{%
    \begin{quote}
        $\blacktriangleright$~#1
    \end{quote}
}

% Définir la fonction pour créer un encadrement de texte
\newcommand{\important}[1]{%
    \begin{tcolorbox}[colback=red!10!white,colframe=red!30!black]
        #1
    \end{tcolorbox}
}



\begin{document}

\entete

\vspace{0.5cm}


\textit{Partie Manquante}


\theoreme{}{
    Let $\rho(S) = \max (\lambda(S))$ be the \textit{spectral radius} of the matrix $S$.
    The iteration associated to the matrix $S$ converges for all initial guess $x^{(0)}$ if and only if $\rho(S) < 1$.
}

\remarque{
    Suppose that the matrix $S$ is symmetric and positive definite.
    We have
    $$
    \frac{||e^{(m)}||}{||e^{(0)}||} \leq ||S^{m}|| \leq \rho(S)^{m}
    $$
}

\remarque{
    How many iterations are needed to guarantee the reduction of the error by a factor of $10^{-p}$?
    $$
    m \geq \frac{\log(10^{-p})}{\log(|\rho(S)|)}
    $$
}

\definition{- Convergence factor}{
    The \textit{convergence factor} is defined as the worst factor of reduction of the error in one step : $\rho(S)$.
}

\definition{- Convergence rate}{
    The \textit{convergence rate} highlights the speed of convergence of the method : $-\log_{10}(\rho(S))$.
    The convergence becomes faster, the higher the convergence rate is.
}






\end{document}
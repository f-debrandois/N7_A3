\documentclass[12pt,a4paper]{article}
\usepackage[utf8]{inputenc} 
\usepackage[T1]{fontenc}		       
\usepackage{lmodern}			       
\usepackage{babel} 
\usepackage{amsmath}
\usepackage{amsfonts}
\usepackage{amssymb}
\usepackage{graphicx}
\usepackage{xcolor}
\usepackage{mathtools}
\usepackage{fancyhdr}
\usepackage{enumitem}
\usepackage{tcolorbox}
\usepackage{stmaryrd}
\usepackage{dsfont}
\usepackage{pgf, tikz}
\usetikzlibrary{shapes.misc}
\usepackage[linesnumbered,ruled,vlined]{algorithm2e}
\usepackage[text={15cm,24.5cm},centering]{geometry}


% Définir le texte affiché en fin de page
\pagestyle{fancy}
\fancyhf{}  % Clear the default headers and footers
\rfoot{\hrule
    \vspace{0.3cm}
    \noindent\textsf{Félix de Brandois}
    \hfill \thepage
}
\renewcommand{\headrulewidth}{0pt}

% Style de l'entete
\newcommand{\entete}{
    \noindent\textit{ModIA S10} \hfill {2024-2025} \\
    {Apprentissage sous contraintes physiques} \hfill {TD} \\
    \begin{center}
        \includegraphics[height=0.8cm]{src/inp_n7.png}
    \end{center}
    \begin{center}
        \textbf{\LARGE TD: Basic knowledge}
    \end{center}
}


\begin{document}

\entete

\section{Matrix Calculus}

\noindent\textbf{Question 1 :}
Let $A$ and $X$ be two real-valued matrices, computes
$$
\frac{\partial \text{Tr}(XA)}{\partial X}
$$

\color{blue}

\noindent\textbf{Answer :}
\begin{align*}
    \frac{\partial \text{Tr}(XA)}{\partial X} &= \left(\frac{\partial \text{Tr}(XA)}{\partial X_{ij}}\right)_{1\leq i,j\leq d} \\
    &= \left(\frac{\partial}{\partial X_{ij}}\sum_{k}\left[XA\right]_{kk}\right)_{1\leq i,j\leq d} \\
    &= \left(\frac{\partial}{\partial X_{ij}}\sum_{k} \sum_{l}X_{kl}A_{lk}\right)_{1\leq i,j\leq d} \\
    &= \left(\frac{\partial}{\partial X_{ij}}\sum_{k} \sum_{l}X_{kl}A_{lk}\right)_{1\leq i,j\leq d} \\
    &= \left(A_{ji}\right)_{1\leq i,j\leq d} \\
    &= A^T
\end{align*}

\color{black}

\noindent\textbf{Question 2 :}
Let $X$ be an invertible real-valued matrix, computes
$$
\frac{\partial \text{det}(X)}{\partial X}
$$

\color{blue}

\begin{align*}
    \frac{\partial \text{det}(X)}{\partial X} &= \left(\frac{\partial \text{det}(X)}{\partial X_{ij}}\right)_{1\leq i,j\leq d} \\
    &= \left(\frac{\partial}{\partial X_{ij}}\text{det}(X)\right)_{1\leq i,j\leq d} \\
    &= \left(\text{det}(X) \text{tr}\left(X^{-1} \frac{\partial X}{\partial X_{ij}}\right)\right)_{1\leq i,j\leq d} \\
    &= \left(\text{det}(X) \text{tr}\left(X^{-1} E_{ij}\right)\right)_{1\leq i,j\leq d} \\
    &= \left(\text{det}(X) \text{tr}\left(X^{-1} E_{ij}\right)\right)_{1\leq i,j\leq d} \\
    &= \left(\text{det}(X) \text{tr}\left(X^{-1} E_{ij}\right)\right)_{1\leq i,j\leq d} \\
    &= \left(\text{det}(X) \text{tr}\left(X^{-1} E_{ij}\right)\right)_{1\leq i,j\leq d} \\
    &= \left(\text{det}(X) \text{tr}\left(X^{-1} E_{ij}\right)\right)_{1\leq i,j\leq d} \\
    &= \left(\text{det}(X) \text{tr}\left(X^{-1} E_{ij}\right)\right)_{1\leq i,j\leq d} \\
    &= \left(\text{det}(X) \text{tr}\left(X^{-1} E_{ij}\right)\right)_{1\leq i,j\leq d} \\
    &= \left(\text{det}(X) \text{tr}\left(X^{-1} E_{ij}\right)\right)_{1\leq i,j\leq d} \\
    &= \left(\text{det}(X) \text{tr}\left
    (X^{-1} E_{ij}\right)\right)_{1\leq i,j\leq d} \\
    &= \left(\text{det}(X) \text{tr}\left(X^{-1} E_{ij}\right)\right)_{1\leq i,j\leq d} \\
    &= \left(\text{det}(X) \text{tr}\left(X^{-1} E_{ij}\right)\right)_{1\leq i,j\leq d} \\
\end{align*}

\end{document}
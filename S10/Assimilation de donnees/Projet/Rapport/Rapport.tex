\documentclass[12pt,a4paper]{article}

\usepackage[T1]{fontenc}
\usepackage[utf8]{inputenc} % Use UTF-8 encoding for input
\usepackage[french]{babel}

\usepackage{lmodern}	
\usepackage{amsmath}
\usepackage{amsfonts}
\usepackage{amssymb}
\usepackage{graphicx}
\usepackage{xcolor}
\usepackage{mathtools}
\usepackage{fancyhdr}
\usepackage{enumitem}
\usepackage{tcolorbox}
\usepackage{colortbl}
\usepackage{multirow}
\usepackage{stmaryrd}
\usepackage{dsfont}
\usepackage{tikz}
\usepackage{hyperref}
\usepackage{subcaption}
\usepackage[upgreek]{txgreeks}
\usepackage{algpseudocode}
\usepackage{algorithm}
\usepackage[text={15cm,24.5cm},centering]{geometry}


% Définir le texte affiché en fin de page
\pagestyle{fancy}
\fancyhf{}  % Clear the default headers and footers
\rfoot{\hrule
    \vspace{0.3cm}
    \noindent\textsf{Assimilation de données}
    \hfill \thepage
}
\renewcommand{\headrulewidth}{0pt}

\title{\vspace{4cm}
        Rapport \\
        \vspace{1cm} \textbf{Estimation de la bathymétrie des rivières à partir de mesures de surface} \\ 
        \vspace{4cm} 
}

\author{\textit{Réalisé par} \vspace{0.5cm}\\
        \textbf{Félix Foucher de Brandois}
}
        
\date{\vfill
        \textit{ENSEEIHT} - 
        \textit{Formation ModIA, 5$^{e}$ année}
        \hfill
        \textit{2024-2025} \\
        \vspace{1cm}
}


\begin{document}

\begin{figure}[t]
    \centering
    \includegraphics[width=7cm]{src/inp_n7.png}
    \hfill
    \includegraphics[width=5.5cm]{src/insa_toulouse.png}
\end{figure}


\maketitle
\thispagestyle{empty}

\newpage


\section{Introduction}

L'objectif de ce rapport est de présenter les résultats d'une étude sur l'estimation de la bathymétrie des rivières à partir de mesures de surface.
Le modèle d'écoulement considéré est le modèle semi-linéarisé, décrit par l'équation :
\begin{equation}
    -\Lambda_{ref}(b)(x)\frac{\partial^{2}H}{\partial x^{2}}(x) + \frac{\partial H}{\partial x}(x) = \frac{\partial b}{\partial x}(x) \quad \forall x \in [0,L]
\end{equation}
avec les conditions aux limites de Dirichlet, et :
$$
\Lambda_{ref}(b) \equiv \Lambda(H_{ref}, b) = \frac{3}{10} \frac{(H_{ref}(x) - b(x))}{|\partial_{x}H_{ref}(x)|}
$$
où $H_{ref}$ est une élévation de surface de référence.
En présence d'observations, on définit : $H_{ref}(x) = H_{obs}(x)$. \\

Le problème inverse consiste à retrouver la bathymétrie $b(x)$ à partir d’observations partielles et potentiellement bruitées de la surface $H_{obs}(x)$.



\section{Analyses des simulations directes}
\subsection{Organisation du code}
Le code fourni est structuré en plusieurs modules :
\begin{itemize}
    \item \textbf{\texttt{main.py}} : Point d'entrée principal. Gère la génération des cas de test, l'exécution du modèle direct et le processus VDA.
    \item \textbf{\texttt{class\_vda.py}} : Implémente la classe \texttt{vda\_river} qui contient les solveurs du modèle direct, adjoint, et les méthodes d'optimisation.
    \item \textbf{\texttt{generate\_case.py}} : Génère des bathymétries synthétiques et définit les conditions aux limites.
    \item \textbf{\texttt{plots.py}} : Produit les visualisations des résultats.
\end{itemize}


La figure \ref{fig:direct_model} présente le résultat de la simulation directe pour différentes bathymétries.

\begin{figure}[H]
    \centering
    \includegraphics[width=0.8\textwidth]{src/direct_model.png}
    \caption{Résultats de la simulation directe pour différentes bathymétries.}
    \label{fig:direct_model}
\end{figure}

lqksjdhfqlkjfh

\subsection{Sensibilité du modèle à la "signature" de surface}

La \textit{signature} de surface désigne 


\section{Problème inverse : Inférence de la bathymétrie}
\subsection{Analyse mathématique de la (non-)unicité de la solution}

Le problème inverse est mal posé car il admet une infinité de solutions pour $b(x)$.
Cependant, si une valeur de $b(x_0)$ est fixée à un point $x_0$, alors la solution devient unique (démontré dans le document supplémentaire).


\subsection{Résolution complète du problème inverse}


\end{document}
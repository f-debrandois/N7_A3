\documentclass[12pt,a4paper]{article}

\usepackage[T1]{fontenc}
\usepackage[utf8]{inputenc} % Use UTF-8 encoding for input
\usepackage[french]{babel}

\usepackage{lmodern}	
\usepackage{amsmath}
\usepackage{amsfonts}
\usepackage{amssymb}
\usepackage{graphicx}
\usepackage{xcolor}
\usepackage{mathtools}
\usepackage{fancyhdr}
\usepackage{enumitem}
\usepackage{tcolorbox}
\usepackage{colortbl}
\usepackage{multirow}
\usepackage{stmaryrd}
\usepackage{dsfont}
\usepackage{tikz}
\usepackage{hyperref}
\usepackage{subcaption}
\usepackage[upgreek]{txgreeks}
\usepackage{algpseudocode}
\usepackage{algorithm}
\usepackage[text={15cm,24.5cm},centering]{geometry}


% Définir le texte affiché en fin de page
\pagestyle{fancy}
\fancyhf{}  % Clear the default headers and footers
\rfoot{\hrule
    \vspace{0.3cm}
    \noindent\textsf{Assimilation de données}
    \hfill \thepage
}
\renewcommand{\headrulewidth}{0pt}

\title{\vspace{4cm}
        Rapport \\
        \vspace{1cm} \textbf{Estimation de la bathymétrie des rivières à partir de mesures de surface} \\ 
        \vspace{4cm} 
}

\author{\textit{Réalisé par} \vspace{0.5cm}\\
        \textbf{Félix Foucher de Brandois}
}
        
\date{\vfill
        \textit{ENSEEIHT} - 
        \textit{Formation ModIA, 5$^{e}$ année}
        \hfill
        \textit{2024-2025} \\
        \vspace{1cm}
}


\begin{document}

\begin{figure}[t]
    \centering
    \includegraphics[width=7cm]{src/inp_n7.png}
    \hfill
    \includegraphics[width=5.5cm]{src/insa_toulouse.png}
\end{figure}


\maketitle
\thispagestyle{empty}

\newpage


\section{Introduction}

L'objectif de ce projet est d'étudier 

L’objectif de ce projet est d’estimer la bathymétrie $b(x)$ (profondeur) des rivières à partir de mesures de la surface de l'eau $H^{\text{obs}}(x)$, en utilisant une approche d’assimilation de données variationnelle (VDA).
On modélise le système par un modèle d'écoulement 




\end{document}
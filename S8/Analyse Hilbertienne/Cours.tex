\documentclass[12pt,a4paper]{article}
\usepackage[utf8]{inputenc} 
\usepackage[T1]{fontenc}		       
\usepackage{lmodern}			       
\usepackage{babel} 
\usepackage{amsmath}
\usepackage{amsfonts}
\usepackage{amssymb}
\usepackage{graphicx}
\usepackage{xcolor}
\usepackage{mathtools}
\usepackage{fancyhdr}
\usepackage{enumitem}
\usepackage{tcolorbox}
\usepackage{stmaryrd}
\usepackage{dsfont}
\usepackage[linesnumbered,ruled,vlined]{algorithm2e}
\usepackage[text={15cm,24.5cm},centering]{geometry}

% Définir le texte affiché en fin de page
\pagestyle{fancy}
\fancyhf{}  % Clear the default headers and footers
\rfoot{\hrule
    \vspace{0.3cm}
    \noindent\textsf{Félix de Brandois}
    \hfill \thepage
}
\renewcommand{\headrulewidth}{0pt}

% Style de l'entete
\newcommand{\entete}{
    \noindent\textbf{INSA - ModIA, 4$^e$ année.}
    \hfill \textbf{Années 2023-2024}
    
    \begin{center}
        \textbf{\LARGE Analyse Hilbertienne}
    \end{center}
}


% Définir la fonction pour créer une boîte de propriété
\newcommand{\propriete}[2]{%
    \begin{tcolorbox}[colback=white,colframe=green!25!white,title=\textbf{Propriété #1}, coltitle=black]
        #2
    \end{tcolorbox}
}

% Définir la fonction pour créer une boîte de définition
\newcommand{\definition}[2]{%
    \begin{tcolorbox}[colback=white,colframe=blue!25!white,title=\textbf{Définition #1}, coltitle=black]
        #2
    \end{tcolorbox}
}

% Définir la fonction pour créer une boîte de théorème
\newcommand{\theoreme}[2]{%
    \begin{tcolorbox}[colback=white,colframe=red!25!white,title=\textbf{Théorème #1}, coltitle=black]
        #2
    \end{tcolorbox}
}

% Définir la fonction pour créer une boîte de remarque
\definecolor{customRed}{RGB}{150, 30, 30}
\newcommand{\remarque}[1]{%
    \leftline{\noindent
    \textcolor{customRed}{\vrule width 3pt}\hspace{10pt}%
    \parbox{0.9\textwidth}{%
        \textbf{Remarque :}
        #1
    }}
    \vspace{10pt}
}

% Définir la fonction pour créer une boîte de preuve
\newcommand{\preuve}[1]{%
    \begin{quote}
        $\blacktriangleright$~#1
    \end{quote}
}



\begin{document}

\entete

\vspace{0.5cm}

\section{Rappels sur les espaces de Hilbert}

\subsection{Espace de Hilbert}

\definition{- Produit scalaire}{
    Soit $E$ un espace vectoriel $(\mathbb{K} = \mathbb{R}$ ou $\mathbb{C})$.\\
    On appelle un \textit{produit scalaire} sur $E$, toute application $E \times E \rightarrow \mathbb{K}$, notée $(x, y) \mapsto \langle x, y \rangle$, vérifiant les propriétés suivantes :
    \begin{itemize}
        \item $\forall x, y \in E, x \mapsto \langle x, y \rangle$ et $y \mapsto \langle x, y \rangle$ sont linéaires
        \item $\forall x, y \in E, \langle x, y \rangle = \overline{\langle y, x \rangle}$ (symétrie conjuguée)
        \item $\forall x \in E, \langle x, x \rangle \geq 0$
        \item $\forall x \in E, \langle x, x \rangle = 0 \Leftrightarrow x = 0$
    \end{itemize}
}

\definition{- Norme}{
    Soit $E$ un espace vectoriel muni d'un produit scalaire $\langle \cdot, \cdot \rangle$.\\
    On appelle \textit{norme} associée à ce produit scalaire, l'application $E \rightarrow \mathbb{R}^+$, notée $x \mapsto \|x\|$, définie par $\|x\| = \sqrt{\langle x, x \rangle}$.
}

\textbf{Exemple :}
\begin{itemize}
    \item $E = \mathbb{C}^d$ muni du produit scalaire $\langle x, y \rangle = \sum_{i=1}^d x_i \overline{y_i}$
    \item $E = \mathbb{R}^d$ muni du produit scalaire $\langle x, y \rangle = \sum_{i=1}^d x_i y_i$
    \item $E = \mathcal{C}([0, 1], \mathbb{C})$ muni du produit scalaire $\langle f, g \rangle = \int_0^1 f(t) \overline{g(t)} dt$
\end{itemize}

\remarque{
    $(E, ||\cdot||)$ est appelé un \textit{espace préhilbertien}.
}

\propriete{- Inégalité de Cauchy-Schwarz}{
    Soit $(E, ||\cdot||)$ un espace préhilbertien.\\
    Alors $\forall x, y \in E, |\langle x, y \rangle| \leq \|x\| \cdot \|y\|$.
}

\preuve{
    \textit{Texte Manquant}
}

\propriete{}{
    \begin{itemize}
        \item $2 \text{Re}(\langle x, y \rangle) = ||x + y||^2 - ||x||^2 - ||y||^2$ (identité de polarisation)
        \item $\|x + y\|^2 + \|x - y\|^2 = 2\|x\|^2 + 2\|y\|^2$ (identité de Parallelogramme)
    \end{itemize}
}

\subsection{Orthogonalité}

\definition{- Orthogonalité}{
    \begin{itemize}
        \item Soit $(x, y) \in E^2$.\\
        On dit que $x$ et $y$ sont \textit{orthogonaux} si $\langle x, y \rangle = 0$.
        \item Soit $A \subset E$.\\
        L'orthogonal de $A$ est l'ensemble $A^\perp = \{x \in E \, | \, \forall y \in A, \langle x, y \rangle = 0\}$.
    \end{itemize}
    }

\remarque{
    \begin{itemize}
        \item $A^\perp$ est un sous-espace vectoriel de $E$.
        \item Si $A = E$, alors $A^\perp = \{0\}$.
        \item $A^\perp = \text{Vect}(A)^\perp$.
    \end{itemize}
}

\textit{Texte Manquant}

\propriete{}{
    Soit $H$ un espace de Hilbert, et $(u_n)$ une suite d'éléments de $H$.\\
    Si la suite $(u_n)$ est composée d'éléments deux à deux orthogonaux et si $\sum ||u_n||^2$ converge, alors la série $\sum u_n$ converge dans $H$.
}

\preuve{
    $\sum |u_n|$ converge $\Leftrightarrow (T_n) = \sum_{k=0}^n ||u_k||$ converge.\\
    $\sum u_n$ converge $\Leftrightarrow (S_n) = \sum_{k=0}^n u_k$ converge.\\
    Si $E$ complet, alors $||S_n - S_m|| = ||\sum_{k=m+1}^n u_k|| \leq \sum_{k=m+1}^n ||u_k|| \leq T_n - T_m$.\\
    Donc $(T_n)$ converge $\Rightarrow (T_n)$ de Cauchy $\Rightarrow (S_n)$ de Cauchy dans un complet $\Rightarrow (S_n)$ converge.
}

\subsection{Projection sur un convexe}

\theoreme{}{
    Soit $H$ un espace de Hilbert, et $C$ un convexe fermé non-vide de $H$.\\
    Pour tout $f \in H$, il existe un unique point $p \in C$, appelé \textit{projection de $f$ sur $C$}, tel que la distance entre $f$ et $p$ soit minimale.\\
    Ce point est caractérisé par : $\forall h \in C, \text{Re}(\langle f - p, h - p \rangle) \leq 0$.
}

\remarque{
    $d(f, C) = \underset{h \in C}{\text{inf }} ||f - h|| = ||f - \Pi_C(f)||$.
}

\preuve{
    \underline{Unicité :} Soient $g_1, g_2 \in C$ tels que $||f - g_1|| = ||f - g_2|| = d(f, C)$.\\
    \textit{Texte Manquant}
}




\end{document}
\documentclass[12pt,a4paper]{article}
\usepackage[utf8]{inputenc} 
\usepackage[T1]{fontenc}		       
\usepackage{lmodern}			       
\usepackage{babel} 
\usepackage{amsmath}
\usepackage{amsfonts}
\usepackage{amssymb}
\usepackage{graphicx}
\usepackage{xcolor}
\usepackage{mathtools}
\usepackage{fancyhdr}
\usepackage{enumitem}
\usepackage{tcolorbox}
\usepackage{algorithm2e}
\usepackage[text={15cm,24.5cm},centering]{geometry}

% Définir le texte affiché en fin de page
\pagestyle{fancy}
\fancyhf{}  % Clear the default headers and footers
\rfoot{\hrule
    \vspace{0.3cm}
    \noindent\textsf{Félix de Brandois}
    \hfill \thepage
}
\renewcommand{\headrulewidth}{0pt}

% Style de l'entete
\newcommand{\entete}{
    \noindent\textbf{INSA - ModIA, 4$^e$ année.}
    \hfill \textbf{Années 2023-2024}
    
    \begin{center}
        \textbf{\LARGE Projet d'EDP}
    \end{center}
}



\begin{document}

\entete

\vspace{0.5cm}


\section*{Approximation spatiale du Laplacien}

\begin{enumerate}
    \item \dots
    \item \dots
    

    \item \begin{itemize}
        \item Intérieur : $\mu \frac{-u_{i+1, j} + 2u_{i,j} - u_{i-1,j}}{h_x^2} + \mu \frac{-u_{i, j+1} + 2u_{i,j} - u_{i,j-1}}{h_y^2} = f(x_i, y_j)$
        \item Côté gauche : $x = 0 :(i = 0, j \in [0, N_{y+1}]) \quad u_{0, j} = 0$
        \item Côté droit : $x = a :(i = N_{x+1}, j \in [0, N_{y+1}]) \quad u_{N_{x+1}, j} = 0$
        \item Côté haut : $y = b :(i \in [0, N_{x+1}], j = N_{y+1}) \quad u_{i, N_{y+1}} = 0$
        \item Côté bas : $y = 0 :(i \in [0, N_{x+1}], j = 0) \quad u_{i, 0} = 0$
    \end{itemize}

    On pose sous forme matricielle : $AU = F$ avec $u_{i,j}$ l'approximation par le schéma de $u(x_i, y_j)$ et :\\

    $U = \begin{pmatrix}
        u_{0,0} \\
        u_{1,0} \\
        \vdots \\
        u_{N_x+1,0} \\
        u_{0,1} \\
        \vdots \\
        u_{N_x+1,N_y+1}
    \end{pmatrix} = \begin{pmatrix}
        U_0 \\
        U_1 \\
        \vdots \\
        U_{N_y+1}
        \end{pmatrix} \in \mathbb{R}^{(N_x+2)(N_y+2)}$ (vecteur colonne)\\

    $F = \begin{pmatrix}
        f(x_0, y_0) \\
        f(x_1, y_0) \\
        \vdots \\
        f(x_{N_x+1}, 0) \\
        f(x_0, y_1) \\
        \vdots \\
        f(x_{N_x+1}, y_{N_y+1})
    \end{pmatrix} \in \mathbb{R}^{(N_x+2)(N_y+2)}$ (vecteur colonne)\\

    On a donc :
    \begin{itemize}
        \item $A = \begin{pmatrix}
            I_d & 0 & 0 & \dots & 0 \\
            C & B & C & \ddots & \vdots \\
            0 & \ddots & \ddots & \ddots & 0 \\
            \vdots & \ddots & C & B & C \\
            0 & \dots & 0 & 0 & I_d
        \end{pmatrix} \in \mathbb{R}^{(N_x+2)(N_y+2) \times (N_x+2)(N_y+2)}$ (matrice creuse)

        \item $B = \begin{pmatrix}
            1 & 0 & \dots & 0 \\
            -\frac{\mu}{h_x^2} & \frac{2\mu}{h_x^2} + \frac{2\mu}{h_y^2} & -\frac{\mu}{h_x^2} & \ddots & \vdots \\
            0 & \ddots & \ddots & \ddots & 0 \\
            \vdots & \ddots & -\frac{\mu}{h_x^2} & \frac{2\mu}{h_x^2} + \frac{2\mu}{h_y^2} & -\frac{\mu}{h_x^2} \\
            0 & \dots & 0 & 0 & 1
        \end{pmatrix} \in \mathbb{R}^{(N_x+2) \times (N_y+2)}$

        \item $C = \begin{pmatrix}
            0 & 0 & \dots & 0 \\
            0 & -\frac{\mu}{h_y^2} & 0 & \ddots & \vdots \\
            0 & \ddots & \ddots & \ddots & 0 \\
            \vdots & \ddots & 0 & -\frac{\mu}{h_y^2} & 0 \\
            0 & \dots & 0 & 0 & 0
        \end{pmatrix} \in \mathbb{R}^{(N_x+2) \times (N_y+2)}$

    \end{itemize}





\end{enumerate}




\end{document}